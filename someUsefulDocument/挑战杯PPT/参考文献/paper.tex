\section{参考文献}
% 
\begin{frame}{参考文献}
\begin{thebibliography}{99}
\small
\bibitem{ref1}[1] Li Y, Wang J, Huang Z, et al. Physics-informed meta learning for machining tool wear prediction[J]. Journal of Manufacturing Systems, 2022, 62: 17-27.
\bibitem{ref2}[2] Gao K, Xu X, Jiao S. Measurement and prediction of wear volume of the tool in nonlinear degradation process based on multi-sensor information fusion[J]. Engineering Failure Analysis, 2022, 136: 106164.
\bibitem{ref3}[3] Obitko M, Jirkovský V. Bezdíček J. Big data challenges in industrial automation[M]. Springer Berlin Heidelberg, 2013, 305-316. 
\bibitem{ref4}[4] Duan J, Zhang X, Shi T. A hybrid attention-based paralleled deep learning model for tool wear prediction[J]. Expert Systems with Applications, 2023, 211: 118548.
\bibitem{ref5}[5] Parwal V, Rout B K. Machine learning based approach for process supervision to predict tool wear during machining[J]. Procedia CIRP, 2021, 98: 133-138.
\bibitem{ref6}[6] Cheng M, Jiao L, Yan P, et al. Intelligent tool wear monitoring and multi-step prediction based on deep learning model[J]. Journal of Manufacturing Systems, 2022, 62: 286-300.
\bibitem{ref7}[7]周成鹏. 基于深度学习的刀具磨损量预测方法[D].中国科学院大学(中国科学院深圳先进技术研究院),2022.DOI:10.27822/d.cnki.gszxj.2022.000045.
\bibitem{ref8}[8]张博闻.基于深度学习的铣刀磨损状态识别及预测[D].哈尔滨理工大学,2022.DOI:10.27063/d.cnki.ghlgu.2022.000670.
\end{thebibliography}
\end{frame}
% 
% 
\begin{frame}{参考文献}
\begin{thebibliography}{99}
\small
\bibitem{ref9}[9]江城子.考虑产品质量信息的刀具更换策略研究[D].吉林大学,2021.DOI:10.27162/d.cnki.gjlin.2021.005060.
\bibitem{ref10}[10]吴茂坤. 基于可靠性分析的数控组合机床维修时间设计[D].吉林大学,2016.
\bibitem{ref11}[11]苏进发. 基于机器视觉的数控刀具磨损检测[D].厦门理工学院,2022.DOI:10.27866/d.cnki.gxlxy.2022.000040.
\bibitem{ref12}[12]令狐克进.多特征融合的车削刀具磨损状态监测技术研究[D].昆明理工大学,2021. DOI:10.27200/d.cnki.gkmlu.2021.000267.
\bibitem{ref13}[13]丁彦玉.五轴数控加工刀具与工件误差源建模及控制策略研究[D].天津大学,2016.
\bibitem{ref14}[14]王继利,杨兆军,李国发,朱晓翠.基于改进EM算法的多重威布尔可靠性建模[J].吉林大学学报(工学版),2014,44(04):1010-1015.DOI:10.13229/j.cnki.jdxbgxb201404017.
\bibitem{ref15}[15]孙继文,奚立峰,潘尔顺,杜世昌.面向产品尺寸质量的制造系统可靠性建模与分析[J].上海交通大学学报,2008(07):1100-1104.DOI:10.16183/j.cnki.jsjtu.2008.07.015.
\end{thebibliography}
\end{frame}
% 