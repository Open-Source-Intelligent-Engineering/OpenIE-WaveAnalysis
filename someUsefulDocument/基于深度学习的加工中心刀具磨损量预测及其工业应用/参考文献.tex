% 
\begin{thebibliography}{99}
\small
\bibitem{ref1} Li Y, Wang J, Huang Z, et al. Physics-informed meta learning for machining tool wear prediction[J]. Journal of Manufacturing Systems, 2022, 62: 17-27.
% 
\bibitem{ref2} Gao K, Xu X, Jiao S. Measurement and prediction of wear volume of the tool in nonlinear degradation process based on multi-sensor information fusion[J]. Engineering Failure Analysis, 2022, 136: 106164.
% 
\bibitem{ref3} Yang Y, Zhao X, Zhao L. Research on asymmetrical edge tool wear prediction in milling TC4 titanium alloy using deep learning[J]. Measurement, 2022, 203: 111814.
% 
\bibitem{ref4} Duan J, Zhang X, Shi T. A hybrid attention-based paralleled deep learning model for tool wear prediction[J]. Expert Systems with Applications, 2023, 211: 118548.
% 
\bibitem{ref5} Parwal V, Rout B K. Machine learning based approach for process supervision to predict tool wear during machining[J]. Procedia CIRP, 2021, 98: 133-138.
% 
\bibitem{ref6} Cheng M, Jiao L, Yan P, et al. Intelligent tool wear monitoring and multi-step prediction based on deep learning model[J]. Journal of Manufacturing Systems, 2022, 62: 286-300.
% 
\bibitem{ref7} Huang Z, Shao J, Guo W, et al. Hybrid machine learning-enabled multi-information fusion for indirect measurement of tool flank wear in milling[J]. Measurement, 2023, 206: 112255.
% 
\bibitem{ref8} He Z, Shi T, Xuan J, et al. Research on tool wear prediction based on temperature signals and deep learning[J]. Wear, 2021, 478: 203902.
% 
\bibitem{ref9} Soori M, Arezoo B, Dastres R. Machine Learning and Artificial Intelligence in CNC Machine Tools, A Review[J]. Sustainable Manufacturing and Service Economics, 2023: 100009.
% 
\bibitem{ref10} Bagri S, Manwar A, Varghese A, et al. Tool wear and remaining useful life prediction in micro-milling along complex tool paths using neural networks[J]. Journal of Manufacturing Processes, 2021, 71: 679-698.
% 
\bibitem{ref11} Koppert S, Henke C, Trächtler A, et al. Tool Wear Monitoring of a Tree Log Bandsaw using a Deep Convolutional Neural Network on challenging data[J]. IFAC-PapersOnLine, 2022, 55(2): 554-560.
% 
\bibitem{ref12} Thanki R. A deep neural network and machine learning approach for retinal fundus image classification[J]. Healthcare Analytics, 2023, 3: 100140.
% 
\bibitem{ref13} Pérez-Sala L, Curado M, Tortosa L, et al. Deep learning model of convolutional neural networks powered by a genetic algorithm for prevention of traffic accidents severity[J]. Chaos, Solitons \& Fractals, 2023, 169: 113245.
% 
\bibitem{ref14} Elalem Y K, Maier S, Seifert R W. A machine learning-based framework for forecasting sales of new products with short life cycles using deep neural networks[J]. International Journal of Forecasting, 2022.
% 
\bibitem{ref15} Meyer J G. Deep learning neural network tools for proteomics[J]. Cell Reports Methods, 2021, 1(2): 100003.
% 
\bibitem{ref16} Cheng M, Jiao L, Yan P, et al. Prediction and evaluation of surface roughness with hybrid kernel extreme learning machine and monitored tool wear[J]. Journal of Manufacturing Processes, 2022, 84: 1541-1556.
% 
\bibitem{ref17} Wang D, Hong R, Lin X. A method for predicting hobbing tool wear based on CNC real-time monitoring data and deep learning[J]. Precision Engineering, 2021, 72: 847-857.
% 
% 
\bibitem{ref18} Zhang C, Zhou G, Li J, et al. A multi-access edge computing enabled framework for the construction of a knowledge-sharing intelligent machine tool swarm in Industry 4.0[J]. Journal of Manufacturing Systems, 2023, 66: 56-70.
% 
\bibitem{ref19} Huang Z, Wiesch M, Fey M, et al. Edge computing-based virtual measuring machine for process-parallel prediction of workpiece quality in metal cutting[J]. Procedia CIRP, 2022, 107: 363-368.
% 
\bibitem{ref20} Zhang J, Deng C, Zheng P, et al. Development of an edge computing-based cyber-physical machine tool[J]. Robotics and Computer-Integrated Manufacturing, 2021, 67: 102042.
% 
% 
% 
\bibitem{ref21}范磊.云制造环境下的车间资源虚拟可视化设计与实现[D].电子科技大学,2020. 
\bibitem{ref22} 龚信.“十四五”智能制造发展规划[N].中国工业报,2021-12-29(003). 
\bibitem{ref23} Obitko M, Jirkovský V. Bezdíček J. Big data challenges in industrial automation[M]. Springer Berlin Heidelberg, 2013, 305-316. 
\bibitem{ref24} 柳天虹.工业大数据时间序列预测方法研究及应用[D].东南大学,2018. 
\bibitem{ref25} 周成鹏,王卫军,侯至丞,冯伟.基于特征提取和长短期记忆神经网络的铣刀磨损量预测[J].控制与信息技术,2021(04):59-65.DOI:10.13889/j.issn.2096-5427.2021.04.100.
\bibitem{ref26} 詹华西,彭超雄.积算加工时间实现数控机床刀具寿命管理[J].制造技术与机床,2008(08):112-114. 
\bibitem{ref27} 周成鹏. 基于深度学习的刀具磨损量预测方法[D].中国科学院大学(中国科学院深圳先进技术研究院),2022.DOI:10.27822/d.cnki.gszxj.2022.000045.
\bibitem{ref28} 张博闻.基于深度学习的铣刀磨损状态识别及预测[D].哈尔滨理工大学,2022.DOI:10.27063/d.cnki.ghlgu.2022.000670.
% 
\bibitem{ref29} 江城子.考虑产品质量信息的刀具更换策略研究[D].吉林大学,2021.DOI:10.27162/d.cnki.gjlin.2021.005060.
\bibitem{ref30} 吴茂坤. 基于可靠性分析的数控组合机床维修时间设计[D].吉林大学,2016.
\bibitem{ref31} 苏进发. 基于机器视觉的数控刀具磨损检测[D].厦门理工学院,2022.DOI:10.27866/d.cnki.gxlxy.2022.000040.
\bibitem{ref32} 令狐克进.多特征融合的车削刀具磨损状态监测技术研究[D].昆明理工大学,2021. DOI:10.27200/d.cnki.gkmlu.2021.000267.
\bibitem{ref33} 丁彦玉.五轴数控加工刀具与工件误差源建模及控制策略研究[D].天津大学,2016.
\bibitem{ref34} 王继利,杨兆军,李国发,朱晓翠.基于改进EM算法的多重威布尔可靠性建模[J].吉林大学学报(工学版),2014,44(04):1010-1015.DOI:10.13229/j.cnki.jdxbgxb201404017.
\bibitem{ref35} 孙继文,奚立峰,潘尔顺,杜世昌.面向产品尺寸质量的制造系统可靠性建模与分析[J].上海交通大学学报,2008(07):1100-1104.DOI:10.16183/j.cnki.jsjtu.2008.07.015.
\end{thebibliography}
% 